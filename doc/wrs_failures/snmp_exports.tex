\section{SNMP exports}
\label{sec:snmp_exports}
This section describes SNMP objects exported by the WR Switch. Objects within
the \texttt{WR\--SWITCH\--MIB} are divided into two categories:
\begin{itemize}
  \item operator/basic objects (section \ref{sec:snmp_exports:basic}) -
    providing basic status of the switch. It should be used by a control system
    operators and people without a deep knowledge of the White Rabbit internals.
    These values report a general status of the device and high level errors.

  \item expert/extended status objects (section \ref{sec:snmp_exports:expert}) -
    can be used by White Rabbit experts for the in-depth diagnosis of the switch
    failures. These values are verbose and should not be used by the operators.
\end{itemize}

\subsection{Operator/basic objects}
\label{sec:snmp_exports:basic}
This section describes the general status MIB objects that are calculated based
on the other SNMP (detailed) exports. Most of the status objects described in
this section can have one of the following values:
\begin{itemize}%[leftmargin=0pt]
  \item \texttt{NA} -- status value was not calculated at all (returned value
    is 0). Something bad has happened.
  \item \texttt{OK} -- status of the particular object is correct.
  \item \texttt{Warning} -- objects used to calculate this value are outside the
    proper values, but problem in not critical enough to report \texttt{Error}.
  \item \texttt{WarningNA} -- at least one of the objects used to calculate the
    status has a value \texttt{NA} or \texttt{WarningNA}.
  \item \texttt{Error} -- error in values used to calculate the particular
    object.
  \item \texttt{FirstRead} -- the value of the object cannot be calculated
    because at least one condition uses deltas between the current and previous
    value. This value should appear only at first SNMP read. Threated as a
    correct value.
  \item \texttt{Bug} -- Something wrong has happened while calculating the
    object. If you see this please report to WR developers.
\end{itemize}

\noindent {\bf General Status objects}:

% SNMP status objects
\printnoidxglossary[type=snmp_status,title=,style=objtree,sort=def]

\newpage
\subsection{Expert/extended status}
\label{sec:snmp_exports:expert}

\noindent {\bf Expert Status}:

% SNMP expert objects
\printnoidxglossary[type=snmp_expert,style=objtree,sort=def]

\subsection{Other's MIB objects}
\label{sec:snmp_exports:others}

\noindent {\bf Objects from other MIBs}:

% other objects
\printnoidxglossary[type=snmp_other,style=objtree,sort=def]

\subsection{Sorted list of MIB objects}
\label{sec:snmp_exports:sorted}

% print alphabetical list
\printnoidxglossary[type=snmp_all,style=tree,sort=letter]

%%%%%%%%%%%%%%%%%%5
%% Other notes
%
% What else should be reported in the future
% Status of Primary Slave port and backup links
% For backup timing links, report parameters from Backup SPLL channels and PTP servo
% What can be reported regarding eRSTP ?
% %	role of the bridge - root/designated
% % port role - root/designated/backup/alternate/disabled
% % number of exchanged BPDUs
%
% * we could use information from RSTP to visualize the topology of network made of switches
% * switches exchange BPDU messages to leard about other switches
% * RFC 2674 - Bridges with priority, multicast pruning and VLAN
