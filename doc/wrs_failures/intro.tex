\section{Introduction}

This document tries to list all possible ways the White Rabbit Switch can
brake. It is my brain dump and should be a starting point to improve SNMP
implementation and alarms (traps) generation. The document also tries to
describe what should be the operator's action for each failure. Whether it's
enough to reboot the switch or if it should be replaced with a new unit.

The document is organized in two parts. First one (section \ref{sec:failures})
tries to list all the possible failures that may disturb synchronization and
Ethernet switching. The structure of each failure description is the following:
\begin{itemize}[leftmargin=0pt]
	\item [] \underline{Mode}: for timing failures, it says which modes are
		affected. Possible values are:
		\begin{itemize}
			\item \emph{Slave} - WR Switch has at least one Slave port synchronized to
				another WR device higher in the timing hierarchy (though it may be also
				Master to other WR/PTP devices lower in the timing hierarchy).
			\item \emph{Grand Master} - WR Switch at the top of the synchronization
				hierarchy. It is synchronized to an external clock (e.g. GPSDO, Cesium)
				and provides timing to other WR/PTP devices.
			\item \emph{Free-Running Master} - WR Switch at the top of the
				synchronization hierarchy. It provides timing to other WR/PTP devices
				but runs from a local oscillator (not synchronized to external atomic
				clock).
		\end{itemize}

	\item [] \underline{Description}: What the problem is about, how important it
		is and what bad may happen if it occurs.
	\item [] \underline{SNMP objects}: Which SNMP objects should be monitored to
		detect the failure. These may be objects from \texttt{WR-SWITCH-MIB} or one
		of the standard MIBs used by the \emph{net-snmp}.
	\item [] \underline{Notes}: Optional comment for SNMP implementation. It may describe current
		implementation of ideas how to implement it in the future
\end{itemize}

Section \ref{sec:snmp_exports} is a documentation for people integrating WR
switch into a control system, operators and WR experts. It describes all
essential SNMP objects exported by the device divided into two groups:
\emph{Operator/basic objects}, \emph{Expert objects}
